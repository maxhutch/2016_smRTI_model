\section{Simple model}

We base our model on the buoyancy-drag models of \cite{Oron2001}:
\begin{equation}
(\rho_1 + \rho_2) \mathcal{V} \ddot{h} = (\rho_2 - \rho_1) \mathcal{V} - C \dot{h}^2 \rho \mathcal{A}
\end{equation}
where $\rho_1$ and $\rho_2$ are the densities of the light and heavy fluid,
$\mathcal{V}$ is the characteristic volume of the bubble
$g$ is the acceleration,
$C$ is a drag-like coefficient, and
$\mathcal{A}$ is the characteristic cross sectional area of the bubble.
Making the Boussinesq approximation, $\rho_1 \approx \rho_2$ yields:
\begin{equation}
\ddot{h} = A g - \frac{C}{2} \dot{h}^2 \frac{\mathcal{A}}{\mathcal{V}}
\end{equation}
In the self-similar regime there is only one length-scale, so $\mathcal{A}/\mathcal{V} \sim 1 / \lambda$.
However, in the single-mode regime that is the focus of this study, the bubbles are elongated, producing two length scales: a span-wise scale $\lambda$ and a stream-wise scale $h$.
Therefore, for the smRTI $\mathcal{A}/\mathcal{V} \sim \frac{1}{h}$ and the model of Oron et. al yields un-bounded velocities:
\begin{equation}
\ddot{h} = A g - \frac{C}{2} \frac{\dot{h}^2}{h}
\end{equation}
Because the strength of the form drag relative to buoyancy decreases at high aspect ratio, we must consider other drag terms, such as skin drag, that grow at least linearly with $h$.

\subsection{Dynamics}

We begin by listing the external forces the bubble experiences.  The first is the buoyant force:
\begin{equation}
F_b = C_0 A g \lambda^2 h,
\end{equation}
where $C_0$ is an unknown coefficient.
The next is the form drag:
\begin{equation}
F_f = C_1 \lambda^2 \dot{h}^2,
\end{equation}
where $C_1$ is similar to a drag coefficient.
The next is the viscous, or skin, drag:
\begin{equation}
F_s = C_2 \nu h \dot{h},
\end{equation}
where $C_2$ is another unknown coefficient and 
$\nu$ is the kinematic vicosity.

To complete the dynamic equation, we must characterize the inertia of the bubble.
The bubble is roughly cylindrical with a height $h$, so we expect an inertial term of the form $\lambda^2 h$.
However, consider the limit of $h \rightarrow 0$.  
Here, streamlines must extend from bubble to spike, which has a characteristic separation $\lambda$ for an inertial term of the form $\lambda^3$.
Therefore, we expect the inertia to be a mix of a term that goes as $\lambda^2 h$ and one that goes as$\lambda^3$:
\begin{equation} \elabel{inertia}
I = C_4 \lambda^2 h + C_3 \lambda^3,
\end{equation}
where $C_4$ and $C_5$ are two more unknown coefficients.

The complete dynamic equation is:
\begin{equation}
\ddot{h} = \frac{C_0 A g \lambda^2 h - C_1 \lambda^2 \dot{h}^2 - C_2 \nu h \dot{h}}{C_3 \lambda^2 h + C_4 \lambda^3}
\end{equation}
Without loss of generality, we can let $C_0 = 1$ and simplify:
\begin{equation}
\ddot{h} = \frac{A g h - C_1 \dot{h}^2 - C_2 \nu (h/\lambda) \dot{h}}{ C_3 h + C_4 \lambda }
\end{equation}
We can non-dimensionalize by defining a dimensionless length and time:
\begin{equation}
z = \frac{h}{\lambda} \qquad \tau = \sqrt{\frac{A g}{\lambda}} t,
\end{equation}
which simplifes:
\begin{equation}
\ddot{z} = \frac{z - C_1 \dot{z}^2 - C_2 \text{Gr}^{-1/2} z \dot{z}}{C_3 z + C_4},
\end{equation}
where $\text{Gr} = A g \lambda^3 \nu^{-2}$ is the Grashof number.


\subsection{Mixing}

As the bubble height grows, the velocity approaches a terminal value specified by the balance between buoyancy and skin drag.
At terminal velocity, the flux of pure fluid into the bubble is bounded.
However, the interfacial mixing continues to grow with the interfacial area, which grows with $h$.
Therefore, for any finite diffusivity, the bubble will ultimately diffuse away.
For this reason, we must include the effects of interficial mixing, at least to the first order.

The quantitiy of mixed fluid $M$ is modeledy by simple diffusion:
\begin{equation}
\dot{M} = D S \frac{1}{\delta}
\end{equation}
where D is the diffusivity,
$S$ is the surface area, and 
$\delta$ is the interface width.
We approximate the scala profile as an error function, yielding an expression for the thickness
\begin{equation}
\delta = \frac{M \sqrt{\pi}}{S}
\end{equation}
In the spirit of the inertial term, \eref{inertia}, we write the surface area as a mix of a terms including and excluding the stream-wise length scale $h$:
\begin{equation}
S = C_5 \lambda h + C_6 \lambda^2
\end{equation}
Togther, this yields:
\begin{equation}
\dot{M} = \frac{1}{\sqrt{\pi}} D (C_5 \lambda h + C_6 \lambda^2)^2 \frac{1}{M}.
\end{equation}
Finally, we couple to the dynamic equation by relating the mixed fluid to the dynamic Atwood number:
\begin{equation} \elabel{effective-atwood}
A = A_0 \left(1 - \frac{M}{C_3 \lambda^2 h  + C_4 \lambda^3} \right)
\end{equation}
We can non-dimensionalize by defining
\begin{equation}
m = \frac{M}{\lambda^3},
\end{equation}
which simplifes:
\begin{equation}
\dot{m} = \frac{1}{\text{Sc} \sqrt{\pi \text{Gr}}} \left(C_5 z + C_6\right)^2 \frac{1}{m},
\end{equation}
where $\text{Sc} = \nu / D$ is the Schmidt number.

\subsection{Limits}
First, consider the limit where $D = 0$, $\nu = 0$, and $h \rightarrow 0$.
The dynamical equation becomes
\begin{equation}
\ddot{h} = \frac{A g }{C_4 \lambda} h
\end{equation}
which matches Rayleigh's original linear stability analysis if 
\begin{equation} \elabel{c4}
C_4 = 1/(2 \pi).
\end{equation}

Next, consider the limit when $\delta \rightarrow 0 $ and $h \rightarrow 0$.
In the linear theory, the Atwood number is rescaled:
\begin{equation}
A = \frac{A_0}{1 + \pi^{-1/2} k \delta} \approx A_0 \left(1 - \frac{k \delta}{\sqrt{\pi}}\right)
\end{equation}
We equate this to \eref{effective-atwood}:
\begin{equation}
\frac{2 \pi \delta}{\lambda \sqrt{\pi}} = \frac{\delta}{\sqrt{\pi}} C_6 \lambda^2 \frac{1}{C_3 \lambda^3}
\end{equation}
Plugging in \eref{c3}, we see that
\begin{equation} \elabel{c6}
C_6 = 1
\end{equation}

Finally, we consider the limit when $h \rightarrow \infty$ and $D = 0$:
The dynamical equation becomes
\begin{equation}
\ddot{h} = \frac{A g - C_2 \nu (1/\lambda) \dot{h}}{C_3}
\end{equation}
which leads to a terminal velocity of:
\begin{equation}
\dot{h} = \frac{A g}{C_2 \nu},
\end{equation}
or a non-dimensional velocity, i.e. Froude number:
\begin{equation}
\text{Fr} = \frac{d z}{d \tau} = \frac{\sqrt{\text{Gr}}}{C_2}.
\end{equation}
The case of extended bubbles and spikes affected only by viscous drag is highly analogous to flow through a square duct.
Therefore, we can use solutions for Poissile flow to estimate $\dot{h}$ and add an approximate constraint on $C_2$.
At the least, this will serve as a good starting guess for $C_2$ for parameter optimization.

\begin{comment}
Next, consider the limit where $D = 0$ and $h \rightarrow 0$.
The dynamical equation becomes:
\begin{equation}
\ddot{h} = \left[\frac{C_0 A g }{C_3 \lambda} - C_2 \nu \dot{h}/\lambda \right] h
\end{equation}
In the spirit of the linear growth model, assume the interface grows as $h(t) = a_0 \cosh[\gamma t] $, with
\begin{equation}
\gamma = \frac{C_0 A g }{C_3 \lambda} - C_2 \nu \dot{h}/\lambda 
\end{equation}
The velocity is given by $\dot{h} = a_0 \gamma \sinh[\gamma t] = \gamma \tanh[\gamma t] h$, yielding:
\begin{equation}
\gamma = \frac{C_0 A g }{C_3 \lambda} - \gamma C_2 \tanh[\gamma t] h \nu /\lambda 
\end{equation}
\end{comment}
