\section{Conclusions}

We have proposed a simple ODE model for the growth of single mode Rayleigh-Taylor bubbles and spikes at low Atwood number.
The model targets an intermediate range of Reynolds numbers and high Peclet number, in which the single mode pertubration grows into an array of coherent bubbles and spikes.
The dynamics of the bubbles are described by a force balance between bouyancy, viscous drag, and form drag.
The buoyant force is scaled by a mixing factor related to the volume faction of mixed fluid within the bubble, which is modeled by diffusion across the bubble's interface.

The viscous drag, which is absent in other buoyancy-drag models, is essential to recovering terminal behavior at high aspect ratios and low diffusivities.  
Without viscous drag or mixing, the buoyant force grows with the aspect ratio while the form drag does not, leading to unbounded bubble velocity.
In practice, as the velocity grows, the bubble interface breaks up leading to enhanced turbulent mixing.

The ODE has 6 parameters, 3 of which can be constrained by limiting cases.
The immiscible linear theory, miscibile linear theory, and Poissile flow limits provide constraints.
The three remaining parameters lend themselves to physical interpretation, providing a guidline for their value.

To validate the simple model under its three constraints, we perform a battery of direct numerical simulations at moderate Reynolds number, high Peclet number, and high aspect ratio.
The simulations provide trajectories for the bubble height and volume of mixed fluid, which are directly compared to the results of the simple model.
Unconstrained, the model is able to reproduce the non-dimensional bubble height to one part in ???.
Constrained by the three limiting cases, the error increases to ???.
This demonstrates the compatability of the constraints with the model.

While the simple model is sufficient to describe coherent, steady bubbles and spikes at low Atwood numbers and high Peclet numbers, few real flows fall within this regime.
In this regard, the simple model proposed here is best seen as a starting point for extensions to more realistic flows.
