\section{Growth stages through late times}

In the spirit of recent analyses~\cite{Ramaprabhu2012, Wei2012}, we try to identify growth stages.

\subsection{Expoential growth}
When the amplitude is small and the interface is thin, the linear theory, simple model, and numerical results all identify exponential growth: $\ddot{h} = \gamma^2 h$.

For low Rayleigh numbers, the bubble stagnates before reaching the top of the computational domain.

For high Rayleigh numbers, the bubble grows until it begins to interact with the finite height domain.
The affected trajectories are therefore incomplete.

Some of the coefficients, specifically $C_7$ and $C_2$, have increasing effects late in the bubble growth.
Therefore, the incomplete trajectories are relatively underconstrained in these parameters, i.e. the trajectory is relatively insensetive to their values.



\subsection{Stagnation}

\subsection{Mixing}

\subsection{Penetration depth}

The penetration depth is the maximum height of the bubble, which is the height of the bubble at the stopping condition of this analysis.
The pentration depth depends strongly on the Rayleigh number and is nearly independent of the Schmidt number, as shown in \fref{depth_scatter}.
Furthermore, the relationship to the Rayleigh number is linear over the cases shown here, as shown in \fref{depth_line}.

