\documentclass[twocolumn,showpacs,preprintnumbers,amsmath,amssymb,floatfix]{revtex4-1}
\usepackage{mysty}
\usepackage{amsmath}
\usepackage{amssymb}
\usepackage{subcaption}
\usepackage{pgfplotstable}
%\usepackage{booktabs}
\usepackage{array}
\usepackage{colortbl}
\usepackage{dashrule}

%\usepackage[caption=false]{subfig}

\begin{document}

\title{Data-driven modeling of the low-Atwood single-mode Rayleigh-Taylor instability}

\author{Maxwell Hutchinson}
\affiliation{The Physics Department, University of Chicago, Chicago IL 60637}
\email{maxhutch@uchicago.edu}

\date{\today}

\begin{abstract}
The Rayleigh-Taylor instability (RTI) pervades classical fluid dynamics and is essential to a diversity of phenomena, e.g. salt fingers, thermonuclear flames, and inertial confinement fusion, but remains poorly understood in dissipative systems.
Recently, the single-mode RTI has shown experimentally and numerically to deviate from established potential flow models when the Atwood number was less than $1/2$.
Attempts to explain the deviation, termed re-acceleration, have been ad-hoc and hindered by a dearth of data at late times and high aspect ratios.
This paper present buoyancy-drag and mixing models that include dissipative terms and match the linear theory.
To inform the model, a numerical experiment is performed, simulating a range of Grashof and Schmidt numbers and reaching bubble heights 34$\times$ the bubble width.
The model coefficients are estimated by physical argument and then fit to the numerical results.
The model error is less than 2\% for the bubble height and 4\% for the volume of mixed fluid.
An attempt is made to interpret variations in the fit parameters with the Rayleigh and Schmidt numbers, where present, but it is hindered by many of the simulations interacting with the boundaries.
Simulations in higher aspect ratio domains would improve the model.
\end{abstract}

\pacs{}
\maketitle

\section{Introduction} \slabel{intro}

The Rayleigh-Taylor instability has been the subject of considerable study since its characterization by Lord Rayleigh in the 19th century~\cite{Rayleigh1883}.
Despite this, many aspects of the non-linear growth remain poorly understood.
Analytic models based on potential flow have been reasonably effective for flows with a large density jump, \ie an Atwood number, $\Delta \rho / \sum \rho$, near unity~\cite{Layzer1955, Layzer1955}.
However, recent experiments at low Atwood number have demonstrated a significant departure from the potential flow limit:
Rayleigh-Taylor bubbles are seen accelerating past the terminal velocity predicted by potential flow models~\cite{Wilkinson2007}.
The acceleration persists beyond times which are experimentally accessible, so numerous efforts have been made to compute the late time flow numerically~\cite{Ramaprabhu2012,Wei2012}.
Ultimately, the goal is to use these numerical experiments to inform a simple model that captures the key features of the flow, as potential flow models do for high Atwood number flows.

This study concerns the dynamics of the single-mode Rayleigh-Taylor instability (smRTI), where the interface between a heavy fluid and a light fluid is perturbed with a single wavelength $\lambda$ and corresponding wavenumber $k$.
If the Atwood number is low, then at early times the interface grows exponentially with a rate given by a linear approximation~\cite{Duff1962}:
\begin{equation} \elabel{duff0}
\gamma = \sqrt{\frac{A g k}{1 + \pi^{-1/2} k \delta} + \nu^2 k^4} - (\nu + D) k^2,
\end{equation}
where $A$ is the Atwood number,
$g$ is the local acceleration,
$\delta$ is the interface thickness,
$\nu$ is the kinematic viscosity, and
$D$ is the diffusivity.
As the amplitude approaches the wavelength, the linear growth saturates.
At unit Atwood number, the non-linear regime is described by potential flow, in which Layzer~\cite{Layzer1955} found the interface approaches a terminal velocity proportional to the root of the acceleration and wavelength: $v_L \sim \sqrt{g \lambda}$:
Potential flow models have been extended $A < 1$, with the most successful model by Goncharov~\cite{Goncharov2002}:
\begin{equation} \elabel{goncharov}
v_G = \frac{1}{\sqrt{\pi}} \sqrt{\frac{A g \lambda}{1+A}}
\end{equation}

Experiments by Wilkinson and Jacobs~\cite{Wilkinson2007} show that after reaching the velocity given by \eref{goncharov}, low Atwood bubbles unexpectedly accelerate a second time.
This is termed `re-acceleration`, with the terminal velocity replaced by a `stagnation velocity`.
Re-acceleration was not present in popular potential flow and buoyancy-drag models, and attempts to capture re-acceleration have been the emphasis of recent model development in the low Atwood number regime.

Experiments have thus far been unable to observe more than the onset of the re-acceleration phase.
Therefore, the community has turned to numerical studies to compute late-time dynamics.
At least two such efforts have been under-taken, one by Ramaprabu \etal~\cite{Ramaprabhu2012} and one by Wei and Livescu~\cite{Wei2012}, leading to slightly different conclusions.
In the study by Ramaprabhu, the flow accelerates to around twice the stagnation velocity and then decelerates back to the stagnation velocity, indicating that re-acceleration may be a transient.
In the study by Wei and Livescu, the flow re-accelerates and then breaks up into many small pockets of buoyant fluid, which themselves continue to accelerate at nearly a fixed rate.

Buoyancy-drag models have been proposed for the multi-mode and single-mode dynamics of the bubble front.
They balance the buoyant force of the bubble with a drag force to predict the front velocity as a function of the characteristic length.
Ramaprabhu proposes an additional forcing term based on the development of a vortex ring at the bubble tip, but relies on observation of the mean vorticity experimentally or in numerical simulations.
There is a need for a predictive model for re-acceleration that relies only on the initial conditions.

In \sref{model}, we propose a simple buoyancy-drag model for the dynamics of the smRTI.
In \sref{exp}, we present a battery of numerical trials.
In \sref{late}, we describe the regimes present in late time bubble trajectories.
In \sref{fit}, we fit the unconstrained model coefficients to the numerical results..
Finally, in \sref{conc}, we assess the current state of smRTI and highlight outstanding questions.



\section{Simple model}

We base our model on the buoyancy-drag models of \cite{Oron2001}:
\begin{equation}
(\rho_1 + \rho_2) \mathcal{V} \ddot{h} = (\rho_2 - \rho_1) \mathcal{V} - C \dot{h}^2 \rho \mathcal{A}
\end{equation}
where $\rho_1$ and $\rho_2$ are the densities of the light and heavy fluid,
$\mathcal{V}$ is the characteristic volume of the bubble
$g$ is the acceleration,
$C$ is a drag-like coefficient, and
$\mathcal{A}$ is the characteristic cross sectional area of the bubble.
Making the Boussinesq approximation, $\rho_1 \approx \rho_2$ yields:
\begin{equation}
\ddot{h} = A g - \frac{C}{2} \dot{h}^2 \frac{\mathcal{A}}{\mathcal{V}}
\end{equation}
In the self-similar regime there is only one length-scale, so $\mathcal{A}/\mathcal{V} \sim 1 / \lambda$.
However, in the single-mode regime that is the focus of this study, the bubbles are elongated, producing two length scales: a span-wise scale $\lambda$ and a stream-wise scale $h$.
Therefore, for the smRTI $\mathcal{A}/\mathcal{V} \sim \frac{1}{h}$ and the model of Oron et. al yields un-bounded velocities:
\begin{equation}
\ddot{h} = A g - \frac{C}{2} \frac{\dot{h}^2}{h}
\end{equation}
Because the strength of the form drag relative to buoyancy decreases at high aspect ratio, we must consider other drag terms, such as skin drag, that grow at least linearly with $h$.

\subsection{Dynamics}

We begin by listing the external forces the bubble experiences.  The first is the buoyant force:
\begin{equation}
F_b = C_0 A g \lambda^2 h,
\end{equation}
where $C_0$ is an unknown coefficient.
The next is the form drag:
\begin{equation}
F_f = C_1 \lambda^2 \dot{h}^2,
\end{equation}
where $C_1$ is similar to a drag coefficient.
The next is the viscous, or skin, drag:
\begin{equation}
F_s = C_2 \nu h \dot{h},
\end{equation}
where $C_2$ is another unknown coefficient and 
$\nu$ is the kinematic vicosity.

To complete the dynamic equation, we must characterize the inertia of the bubble.
The bubble is roughly cylindrical with a height $h$, so we expect an inertial term of the form $\lambda^2 h$.
However, consider the limit of $h \rightarrow 0$.  
Here, streamlines must extend from bubble to spike, which has a characteristic separation $\lambda$ for an inertial term of the form $\lambda^3$.
Therefore, we expect the inertia to be a mix of a term that goes as $\lambda^2 h$ and one that goes as$\lambda^3$:
\begin{equation} \elabel{inertia}
I = C_4 \lambda^2 h + C_3 \lambda^3,
\end{equation}
where $C_4$ and $C_5$ are two more unknown coefficients.

The complete dynamic equation is:
\begin{equation}
\ddot{h} = \frac{C_0 A g \lambda^2 h - C_1 \lambda^2 \dot{h}^2 - C_2 \nu h \dot{h}}{C_3 \lambda^2 h + C_4 \lambda^3}
\end{equation}
Without loss of generality, we can let $C_0 = 1$ and simplify:
\begin{equation}
\ddot{h} = \frac{A g h - C_1 \dot{h}^2 - C_2 \nu (h/\lambda) \dot{h}}{ C_3 h + C_4 \lambda }
\end{equation}
We can non-dimensionalize by defining a dimensionless length and time:
\begin{equation}
z = \frac{h}{\lambda} \qquad \tau = \sqrt{\frac{A g}{\lambda}} t,
\end{equation}
which simplifes:
\begin{equation}
\ddot{z} = \frac{z - C_1 \dot{z}^2 - C_2 \text{Gr}^{-1/2} z \dot{z}}{C_3 z + C_4},
\end{equation}
where $\text{Gr} = A g \lambda^3 \nu^{-2}$ is the Grashof number.


\subsection{Mixing}

As the bubble height grows, the velocity approaches a terminal value specified by the balance between buoyancy and skin drag.
At terminal velocity, the flux of pure fluid into the bubble is bounded.
However, the interfacial mixing continues to grow with the interfacial area, which grows with $h$.
Therefore, for any finite diffusivity, the bubble will ultimately diffuse away.
For this reason, we must include the effects of interficial mixing, at least to the first order.

The quantitiy of mixed fluid $M$ is modeledy by simple diffusion:
\begin{equation}
\dot{M} = D S \frac{1}{\delta}
\end{equation}
where D is the diffusivity,
$S$ is the surface area, and 
$\delta$ is the interface width.
We approximate the scala profile as an error function, yielding an expression for the thickness
\begin{equation}
\delta = \frac{M \sqrt{\pi}}{S}
\end{equation}
In the spirit of the inertial term, \eref{inertia}, we write the surface area as a mix of a terms including and excluding the stream-wise length scale $h$:
\begin{equation}
S = C_5 \lambda h + C_6 \lambda^2
\end{equation}
Togther, this yields:
\begin{equation}
\dot{M} = \frac{1}{\sqrt{\pi}} D (C_5 \lambda h + C_6 \lambda^2)^2 \frac{1}{M}.
\end{equation}
Finally, we couple to the dynamic equation by relating the mixed fluid to the dynamic Atwood number:
\begin{equation} \elabel{effective-atwood}
A = A_0 \left(1 - \frac{M}{C_3 \lambda^2 h  + C_4 \lambda^3} \right)
\end{equation}
We can non-dimensionalize by defining
\begin{equation}
m = \frac{M}{\lambda^3},
\end{equation}
which simplifes:
\begin{equation}
\dot{m} = \frac{1}{\text{Sc} \sqrt{\pi \text{Gr}}} \left(C_5 z + C_6\right)^2 \frac{1}{m},
\end{equation}
where $\text{Sc} = \nu / D$ is the Schmidt number.

\subsection{Limits}
First, consider the limit where $D = 0$, $\nu = 0$, and $h \rightarrow 0$.
The dynamical equation becomes
\begin{equation}
\ddot{h} = \frac{A g }{C_4 \lambda} h
\end{equation}
which matches Rayleigh's original linear stability analysis if 
\begin{equation} \elabel{c4}
C_4 = 1/(2 \pi).
\end{equation}

Next, consider the limit when $\delta \rightarrow 0 $ and $h \rightarrow 0$.
In the linear theory, the Atwood number is rescaled:
\begin{equation}
A = \frac{A_0}{1 + \pi^{-1/2} k \delta} \approx A_0 \left(1 - \frac{k \delta}{\sqrt{\pi}}\right)
\end{equation}
We equate this to \eref{effective-atwood}:
\begin{equation}
\frac{2 \pi \delta}{\lambda \sqrt{\pi}} = \frac{\delta}{\sqrt{\pi}} C_6 \lambda^2 \frac{1}{C_3 \lambda^3}
\end{equation}
Plugging in \eref{c3}, we see that
\begin{equation} \elabel{c6}
C_6 = 1
\end{equation}

Finally, we consider the limit when $h \rightarrow \infty$ and $D = 0$:
The dynamical equation becomes
\begin{equation}
\ddot{h} = \frac{A g - C_2 \nu (1/\lambda) \dot{h}}{C_3}
\end{equation}
which leads to a terminal velocity of:
\begin{equation}
\dot{h} = \frac{A g}{C_2 \nu},
\end{equation}
or a non-dimensional velocity, i.e. Froude number:
\begin{equation}
\text{Fr} = \frac{d z}{d \tau} = \frac{\sqrt{\text{Gr}}}{C_2}.
\end{equation}
The case of extended bubbles and spikes affected only by viscous drag is highly analogous to flow through a square duct.
Therefore, we can use solutions for Poissile flow to estimate $\dot{h}$ and add an approximate constraint on $C_2$.
At the least, this will serve as a good starting guess for $C_2$ for parameter optimization.

\begin{comment}
Next, consider the limit where $D = 0$ and $h \rightarrow 0$.
The dynamical equation becomes:
\begin{equation}
\ddot{h} = \left[\frac{C_0 A g }{C_3 \lambda} - C_2 \nu \dot{h}/\lambda \right] h
\end{equation}
In the spirit of the linear growth model, assume the interface grows as $h(t) = a_0 \cosh[\gamma t] $, with
\begin{equation}
\gamma = \frac{C_0 A g }{C_3 \lambda} - C_2 \nu \dot{h}/\lambda 
\end{equation}
The velocity is given by $\dot{h} = a_0 \gamma \sinh[\gamma t] = \gamma \tanh[\gamma t] h$, yielding:
\begin{equation}
\gamma = \frac{C_0 A g }{C_3 \lambda} - \gamma C_2 \tanh[\gamma t] h \nu /\lambda 
\end{equation}
\end{comment}


\section{Conclusions}

We have proposed a simple ODE model for the growth of single mode Rayleigh-Taylor bubbles and spikes at low Atwood number.
The model targets an intermediate range of Reynolds numbers and high Peclet number, in which the single mode pertubration grows into an array of coherent bubbles and spikes.
The dynamics of the bubbles are described by a force balance between bouyancy, viscous drag, and form drag.
The buoyant force is scaled by a mixing factor related to the volume faction of mixed fluid within the bubble, which is modeled by diffusion across the bubble's interface.

The viscous drag, which is absent in other buoyancy-drag models, is essential to recovering terminal behavior at high aspect ratios and low diffusivities.  
Without viscous drag or mixing, the buoyant force grows with the aspect ratio while the form drag does not, leading to unbounded bubble velocity.
In practice, as the velocity grows, the bubble interface breaks up leading to enhanced turbulent mixing.

The ODE has 6 parameters, 3 of which can be constrained by limiting cases.
The immiscible linear theory, miscibile linear theory, and Poissile flow limits provide constraints.
The three remaining parameters lend themselves to physical interpretation, providing a guidline for their value.

To validate the simple model under its three constraints, we perform a battery of direct numerical simulations at moderate Reynolds number, high Peclet number, and high aspect ratio.
The simulations provide trajectories for the bubble height and volume of mixed fluid, which are directly compared to the results of the simple model.
Unconstrained, the model is able to reproduce the non-dimensional bubble height to one part in ???.
Constrained by the three limiting cases, the error increases to ???.
This demonstrates the compatability of the constraints with the model.

While the simple model is sufficient to describe coherent, steady bubbles and spikes at low Atwood numbers and high Peclet numbers, few real flows fall within this regime.
In this regard, the simple model proposed here is best seen as a starting point for extensions to more realistic flows.





\bibliography{library}

\end{document}

